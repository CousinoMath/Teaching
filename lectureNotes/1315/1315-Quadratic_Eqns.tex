% !TeX encoding = UTF-8
% !TeX spellcheck = en_US
% !TeX root = 1315LectureNotes.tex
\chapter{Quadratic Equations}\label{chap:quadEqns}
\begin{genericFrame}[frametitle={~New Things\hbox{~}}]
    \textbf{\Large\sffamily Definitions}
    \begin{description}[style=nextline]
        \item[Quadratic Equation] An equation which can be put in the %
        form \(\constcolor{a} x^{2} + \constcolor{b} x + %
         \constcolor{c}=0\).
    \end{description}

    \noindent\textbf{\Large\sffamily Procedures}
    \begin{description}[style=nextline]
    	\item[Factoring a Quadratic]
        \item[Completing the Square] A process that allows us to %
         rewrite a quadratic from standard form into vertex form.
        \item[Quadratic Formula]
    \end{description}

	\noindent\textbf{\Large\sffamily Notation}
	\begin{description}
		\item[Standard Form of a Quadratic] \(\constcolor{a} x^{2} +
		 \constcolor{b} x + \constcolor{c}\)
		\item[Vertex Form of a Quadratic] \(\constcolor{a}
		 \parens{x - \constcolor{h}}^{2} + \constcolor{k}\)
	
	\end{description}
\end{genericFrame}
\section{Factoring}
\section{Completing the Square}
\section{Quadratic Formula}
