% !TeX encoding = UTF-8
% !TeX spellcheck = en_US
% !TeX root = 1315LectureNotes.tex
\chapter{Complex Numbers}
\begin{genericFrame}[frametitle={~Important Things\hbox{~}}]
    \textbf{\Large\sffamily Definitions}
    \begin{description}[style=nextline]
    	\item[Real Numbers] All the numbers you know and love/hate 
    	 like \(0\), \(1\), \(7/11\), \(\sqrt{2}\), and \(\pi\).
    	\item[\(i\)] The square root of \(-1\), meaning \(i^{2}=-1\).
    	\item[Complex Numbers] All ``numbers'' of the form 
    	 \(\constcolor{a} + \constcolor{b} i\) where both
    	 \(\constcolor{a}\) and \(\constcolor{b}\) are real numbers.
    	\item[Conjugate of a Complex Number] The conjugate of an
    	 arbitrary complex number, like \(\constcolor{a} + 
    	 \constcolor{b} i\), is \(\constcolor{a} - \constcolor{b} i\).
    	 Simply put, you change the sign on the number in front of 
    	 \(i\).
    \end{description}
    
    \textbf{\noindent\Large\sffamily Procedures}
    \begin{description}[style=nextline]
    	\item[Arithmetic of Complex Numbers] Including when two
    	 complex numbers equal each other as well as how to add, 
    	 subtract, multiply, and divide complex numbers.
    \end{description}
    
    \textbf{\noindent\Large\sffamily Notations}
    \begin{description}[style=nextline]
    	\item[Standard Form of a Complex Number] It is writing a 
    	complex number as \(\constcolor{a} + \constcolor{b} i\).
    \end{description}
\end{genericFrame}
