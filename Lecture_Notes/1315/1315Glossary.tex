\usepackage[xindy]{imakeidx}
\usepackage[toc,xindy,symbols]{glossaries}
\makeglossaries
\makeindex

\newglossaryentry{set}{
    name=set,
    description={A collection of items, which we call elements}
}
\newglossaryentry{element}{
    name=element,
    description={A member, or item within, a set},
    see={set}
}
\newglossaryentry{emptySet}{
    name={empty set},
    description={The set that contains no elements},
    symbol={\ensuremath{\emptyset}},
    see={set,element}
}
\newglossaryentry{rosterMethod}{
    name={roster method},
    description={This method of defining a set amounts to simply
        listing out all the elements of the set withing curly braces,
        \ensuremath{\left\{\right\}}},
    see={set,element}
}
\newglossaryentry{builderMethod}{
    name={builder method},
    description={This method of defining a set describes a membership
        requirement. Every element of the set satisfies the membership
        requirement by making it true, and everything that satisfies
        the requirement is a member. All other items aren't members of
        the set and fail the membership requirement. For example,
        \ensuremath{\left\{ x\mid x\textnormal{ is a prime number}
            \right\}} consists of elements 2, 3, 5, 7, 11, etc},
    see={set,element}
}
\newglossaryentry{realNumbers}{
    symbol={\ensuremath{\mathbf{R}}},
    name={real numbers},
    description={The set of all integers, fractions, and irrational
        numbers like \ensuremath{\sqrt{5}} and \ensuremath{\pi}}
}